\begin{thebibliography}{9}

\bibitem{uri}
T. Berners-Lee, R. Fielding, L. Masinter.,
\textit{Uniform Resource Identifier (URI): Generic Syntax.} \\ 
\texttt{https://tools.ietf.org/pdf/rfc3986.pdf}, 2005

\bibitem{encoding}
S. Josefsson,
\textit{The Base16, Base32, and Base64 Data Encodings} \\ 
\texttt{https://tools.ietf.org/pdf/rfc4648}, SJD, 2006

\bibitem{oneway}
Oded Goldreich,
\textit{Foundations of Cryptography: Volume 1, Basic Tools}  
Cambridge University Press. ISBN 0-521-79172-3., 2001

\bibitem{md5cert}
Arjen Lenstra, Xiaoyun Wang,and Benne de Weger. Colliding
x.509 certificates. Cryptology ePrint Archive, Report 2005/067,
2005.

\bibitem{md5rfc}
R. Rivest,
\textit{The MD5 Message-Digest Algorithm} \\ 
\texttt{https://tools.ietf.org/pdf/rfc1321.pdf}, MIT, 1992

\bibitem{otprfc}
N. Haller, C. Metz, P. Nesser, M. Straw, 
\textit{A One-Time Password System} \\ 
\texttt{https://tools.ietf.org/pdf/rfc2289}, MIT, 1998

\bibitem{hotprfc}
D. M'Raihi, M. Bellare, F. Hoornaert, D. Naccache, O. Ranen, 
\textit{HOTP: An HMAC-Based One-Time Password Algorithm} \\ 
\texttt{https://tools.ietf.org/pdf/rfc4226}, 2005

\bibitem{totprfc}
D. M'Raihi, S. Machani, M. Pei, J. Rydell,
\textit{TOTP: Time-Based One-Time Password Algorithm} \\ 
\texttt{https://tools.ietf.org/pdf/rfc6238}, 2011

\bibitem{pandp}
Ken Shirriff,
\textit{Mining Bitcoin with pencil and paper: 0.67 hashes per day} \\ 
\texttt{http://www.righto.com/2014/09/mining-bitcoin-with-pencil-and-paper.html}, 2014

\bibitem{sha1rfc}
D. Eastlake, 3rd, P. Jones
\textit{US Secure Hash Algorithm 1 (SHA1)} \\ 
\texttt{https://tools.ietf.org/pdf/rfc3174}, 2001

\bibitem{crypto101}
Laurens Van Houtven (lvh),
\textit{Crypto 101} \\ 
\texttt{https://www.crypto101.io}, 2017

\bibitem{keccak}
Keccak Team,
\textit{Strengths of Keccak - Design and security} \\ 
\texttt{https://keccak.team/keccak\_strengths.html}, 2017

\bibitem{cryptoirf}
Niels Ferguson, Bruce Schneier, Todayoshi Kohno
\textit{Cryptography Engineering: Design Principles and Practical Applications}. Wiley Publishing, Inc., 2010

\bibitem{shap}
Marc Stevens, Pierre Karpman, Thomas Peyrin,
\textit{The SHAppening: freestart collisions for SHA-1}. \\
\texttt{https://sites.google.com/site/itstheshappening}, 2015

\bibitem{shatt}
Marc Stevens, Elie Bursztein, Pierre Karpman, Ange Albertini, Yarik Markov
\textit{The first collision for full SHA-1}. \\
\texttt{https://shattered.io/static/shattered.pdf}, CWI Amsterdam, Google Research, 2017

\bibitem{beast}
Thai Duong, Juliano Rizzo, 
\textit{Here Come The $\xor$ Ninjas}. \\
\texttt{https://bug665814.bmoattachments.org/attachment.cgi?id=540839}, 2011

\bibitem{desday}
SciEngines GmbH
\textit{Break DES in less than a single day}. \\
\texttt{https://www.voltage.com/technology/rivyera-from-sciengines/}, 2008

\bibitem{dist}
\textit{DES-III contest}. \\
\texttt{http://www.distributed.net/DES}, 1999

\bibitem{typingaes}
Thomas Ptacek, 
\textit{If You’re Typing the Letters A-E-S Into Your Code You’re Doing It Wrong}. \\
\texttt{https://people.eecs.berkeley.edu/~daw/teaching/cs261-f12/misc/if.html}, 2009

\bibitem{tls}
Serge Vaudenay,
\textit{Security Flaws Induced by CBC Padding Applications to SSL, IPSEC, WTLS...}. \\
\texttt{https://www.iacr.org/cryptodb/archive/2002/EUROCRYPT/2850/2850.pdf}, 2002

\bibitem{hmacmd5}
Mihir Bellare,
\textit{New proofs for NMAC and HMAC: Security without collision-resistance}. \\
\texttt{http://cseweb.ucsd.edu/\textasciitilde mihir/papers/hmac-new.html}, 2006

\end{thebibliography}