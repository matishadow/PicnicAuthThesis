\chapter{Kryptografia w praktyce}

\section{Pojęcia pomocnicze}
Przed przystąpieniem do opisu praktycznych aspektów kryptografii użytych w projekcie, wymagane jest 
wyjaśnienie pojęć wykorzystywanych w mechanizmie haseł jednorazowych, lecz które nie są bezpośrednio 
związane z kryptografią. 

\subsection{Kodowanie transportowe}
Kodowanie transportowe wykorzystywane jest w przypadku, gdy zachodzi potrzeba transferu danych
w środowiskach, które pozwalają na przesyłanie wyłącznie znaków ASCII. \\ \\
Użycie kodowania transportowego jest konieczne w celu zachowania kompatybilności przy pracy z protokołami, 
które przystosowane są do pracy na danych 7-bitowych. W takim przypadku najstarszy bit jest zerowany, co
mogłoby uszkodzić przesyłane dane. W przypadku przesyłania wyłącznie znaków ASCII zerowanie najstarszego bitu
nie jest problemem, gdyż wszystkie znaki w podstawowej tablicy ASCII mają ten bit wyzerowany.\\
Bardziej współczesnym przykładem wykorzystania kodowania transportowego jest osadzanie danych graficznych bezpośrednio w kodzie HTML. Konieczne jest wówczas zakodowanie danych w celu wyeliminowania ryzyka pojawienia się znaków '<' oraz '>', które mogłyby być zinterpretowane jako tagi HTML. \\ \\
Aby ujednolicić implementacje kodowania transportowego został stworzony dokument RFC 4648 \cite{encoding}, w którym opisany jest prawidłowy sposób implementacji oraz to jaki typ kodowania wybrać w zależności od nałożonych wymagań.

\subsubsection{Kodowanie Base64}
Najczęściej spotykanym typem kodowania transportowego jest kodowanie Base64. Kodowanie to konwertuje dowolny ciąg bajtów do postaci ciągu złożonego z małych i wielkich liter, cyfr oraz znaków '+' i '/'. 
Jeżeli po zakodowaniu końcowa część danych jest mniejsza niż 24 bity używany jest także znak '=' jako dopełnienie. \\
Sam proces kodowania polega na pobraniu 24 bitów danych a następnie podzieleniu ich na 4 grupy po 6 bitów. Każda z grup jest interpretowana jako indeks tablicy ustalonego alfabetu Base64.
Dla każdej z grup za pomocą indeksu odczytywany jest znak a następnie dopisywany jest on do ciągu zakodowanego. \\
Istnieje również odmiana kodowania Base64 przystosowana do użycia w przypadku adresów URL czy nazw plików.
W alternatywie tej zamiast znaków '+', '/', które mogłyby zostać błędnie zinterpretowane np w środowisku systemu plików,
używane są znaki '-' oraz '\_'.

\subsubsection{Kodowanie Base32}
W porównaniu do kodowania Base64, dane zakodowane w Base32 są dużo bardziej czytelne dla ludzi.
Właściwość ta spowodowana jest faktem, że w kodowaniu Base32 nie ma znaczenia wielkość liter, dzięki czemu przykładowo
nie ma problemu z rozróżnieniem małej litery 'L' z wielką literą 'I' ('l' oraz 'I'). \\
Alfabet kodowania Base32 składa się z 32 znaków ASCII oraz znaku '=' pełniącego funkcje dopełnienia.
Proces kodowania polega na pobraniu 40 bitów danych a następnie ustawienie ich w osiem 5-bitowych grup. 
Każda z 8 grup interpretowana jest jako jeden ze znaków alfabetu Base32. \\
Podobnie jak przy kodowaniu Base64, wymagane jest tutaj wstawienie dopełnienia w sytuacji, gdy długość ostatniej z grup jest mniejsza od 40 bitów.

\begin{table}[h!]
\centering
\caption{Alfabet w kodowaniu Base32}
\begin{tabular}{ |c c|c c|c c|c c| } 
 \hline
 Indeks & Znak & Indeks & Znak & Indeks & Znak & Indeks & Znak \\ 
 \hline
 0 & A & 9  & J & 18 & S & 27 & 3 \\ 
 1 & B & 10 & K & 19 & T & 28 & 4 \\ 
 2 & C & 11 & L & 20 & U & 29 & 5 \\ 
 3 & D & 12 & M & 21 & V & 30 & 6 \\ 
 4 & E & 13 & N & 22 & W & 31 & 7 \\ 
 5 & F & 14 & O & 23 & X &    &   \\ 
 6 & G & 15 & P & 24 & Y &    &   \\  
 7 & H & 16 & Q & 25 & Z &    &   \\ 
 8 & I & 17 & R & 26 & 2 &    &   \\ 
 \hline
\end{tabular}
\end{table}

\subsection{Czas uniksowy}
Czas uniksowy jest sposobem na reprezentację punktu w czasie, polegającym na mierzeniu sekund, które
upłynęły od daty 1 stycznia 1970 (UTC). W systemach uniksowych zwykle reprezentowany jest w postaci
32-bitowej liczby całkowitej ze znakiem. \\
W przypadku architektur typu serwer-klient wskazane jest synchronizowanie czasu wykorzystując czas~uniksowy,
gdyż nie zależy on od lokalizacji w której jest mierzony. Właściwość ta eliminuje problem synchronizacji 
czasu pomiędzy strefami czasowymi.

\subsection{Ujednolicony identyfikator zasobów}
Ujednolicony identyfikator zasobów (ang. Uniform Resource Identifier, URI) jest ciągiem znaków jednoznacznie identyfikującym dany zasób. \\
Składnia identyfikatora jest wyrażana następująco: \\
\centerline{schemat ":" ścieżka ["?" zapytanie] ["\#" fragment]} 
Warto zauważyć, że składnia ta determinuje schemat (protokół), jaki wykorzystywany jest przy interakcji z identyfikowanym zasobem. \\

Przykłady identyfikatorów:
\begin{itemize}
	\item ftp://randomftp.com/files/file.docx
	\item https://www.randomwebsite.pl/index.html
	\item mailto:jan.nowak@wp.pl
	\item tel:+48-25-123-88
\end{itemize}

Szczegóły dotyczące standardu URI są opisane w dokumencie RFC~3986 \cite{uri}. 

 
\section{Hasło jednorazowe}
Hasło jednorazowe jest jedną z możliwych form uwierzytelnienia przy używaniu uwierzytelnienia wieloetapowego. \\
W odróżnieniu od zwykłego, statycznego hasła, haseł jednorazowych można użyć tylko raz. 
Dzięki czemu przechwycenie przez atakującego, raz użytego hasła, nie daje możliwości uwierzytelnienia. \\
Hasła jednorazowe eliminują również problem zbyt niskiej entropii w hasłach. 
Często zdarza się, że hasła tworzone przez użytkowników są zbyt słabe, przez co nie zapewniają wystarczającego
poziomu bezpieczeństwa. Hasła jednorazowe powstają przy użyciu bezpiecznych generatorów liczb pseudolosowych, 
dzięki czemu zapewniony jest wystarczający poziom entropii (pod warunkiem, że generatory te są w odpowiedni 
sposób używane). \\ \\
Istnieje wiele metod generacji haseł jednorazowych o różnym stopniu bezpieczeństwa. 
Niektóre z tych metod to:
\begin{enumerate}
	\item Użycie łańcucha wyników funkcji skrótu.
	\item Użycie mechanizmu HMAC w połączeniu z synchronizowanym licznikiem.
	\item Użycie mechanizmu HMAC z synchronizacją czasu.
\end{enumerate}
Standard uwierzytelniania za pomocą haseł jednorazowych zdefiniowany jest w dokumencie \textit{RFC 2289} \cite{otprfc}.

\subsection{Hasło oparte o HMAC}
Do generacji haseł tego typu wykorzystywana jest funkcja HMAC. Może być tu użyta dowolna funkcja skrótu, jednak 
musi być ona taka sama po stronie generującej jak i weryfikującej. 
Na wejście funkcji podawane są dwa argumenty, symetryczny klucz oraz licznik. 
Konieczne jest aby obie wartości były zsynchronizowane pomiędzy użytkownikiem a stroną uwierzytelniającą. 
Przy każdej udanej walidacji hasła jednorazowego licznik zwiększany jest o 1. \\
Przykładowo używając funkcji skrótu SHA-1, wynik HMAC-SHA-1 o długości 160 bitów jest obcinany do kilku znakowego hasła. 
Hasło powinno być używane w postaci numerycznej a za minimalną bezpieczną długość hasła uznawane jest sześć cyfr.
Minimalną długością klucza kryptograficznego jest 128 bitów, rekomendowane jest jednak używanie przynajmniej 160-bitowego 
klucza. \\ \\
Proces generacji hasła jednorazowego można przedstawić następująco: \\
$$HOTP(K, C) = Przytnij(\hmac(K, C))$$
Przycięcie przebiega w dwóch etapach. W pierwszym etapie 160-bitowy wynik funkcji HMAC skracany jest do czterech bajtów.
Z 160 bitów pobierane są 4 ostatnie bity. 
Wykorzystywane są one jako indeks, od którego pobieramy 4 bajty z wyniku funkcji HMAC.
Drugi etap polega na obliczeniu reszty z dzielenia z poprzedniego wyniku przez $10^n$, gdzie \textit{n} jest 
liczbą cyfr jaką chcemy ostatecznie otrzymać. Algorytm jak i przykładowa implementacja w języku \textit{Java} znajduje się w dokumencie \textit{RFC~4226}~\cite{hotprfc}.

\subsection{Hasło oparte o czas}
Algorytm generacji haseł opartych o czas jest ulepszoną wersją poprzedniego. 
Ponownie używana jest funkcja HMAC, jednak zamiast licznika wykorzystywany jest czas uniksowy.
Przed podaniem czasu uniksowego do funkcji HMAC jest on dzielony przez okno czasowe czyli 
przez liczbę sekund, jaką dane hasło jest ważne. 
Okno czasowe zwykle jest ustawiane na 30 lub 60 sekund. \\
Wymogi dotyczące klucza kryptograficznego jak również proces przycinania wyniku funkcji HMAC 
do postaci hasła jednorazowego jest identyczny jak w poprzednim algorytmie. \\
W przypadku haseł opartych o czas konieczna jest synchronizacja zegarów pomiędzy stroną generującą hasło
oraz stroną uwierzytelniającą. \\
Dokumentem, w którym opisany jest algorytm jest \textit{RFC 6238} \cite{totprfc}.

\subsection{Porównanie HOTP i TOTP}
Przewagą haseł opartych o czas jest ich krótki czas ważności. 
Ich użycie jest możliwe tylko w z góry określonym oknie czasowym. 
Ważność haseł opartych o licznik jest praktycznie nieograniczona, są ważne aż do momentu ich użycia. \\
W przypadku podejrzenia hasła opartego o czas, wygenerowanego na czyimś telefonie, atakujący nie jest 
w stanie pójść do domu i spróbować użyć wykradzione hasło. 
Przechwycone hasło oparte o licznik jest możliwe do użycia w dowolnej chwili. \\
Drugim aspektem przemawiającym za używanie haseł opartych o czas jest kwestia ataków siłowych.
Ataki siłowe skierowane bezpośrednio na hasło jednorazowe są nieco mniej efektywne, gdyż 
szukane hasło w każdym oknie czasowym jest inne.

\subsection{Porównanie kanałów dostarczania}
\subsubsection{,,Zdrapka''}
Zwykle wysyłane pocztą w postaci dokumentu z wydrukowaną listą haseł lub zdrapki w rozmiarze 
karty bankomatowej. W celu potwierdzenia transakcji użytkownik proszony jest o podanie 
kolejnego z haseł zapisanych na zdrapce lub jedno z haseł ze zdrapki o numerze wybranym
przez serwer w pseudolosowy sposób. \\
Zgubienie takiej karty z hasłami jednorazowymi jest równoznaczne ze złamaniem jednego z etapów 
w uwierzytelnianiu wieloetapowym. Również sam sposób dostarczania zdrapki w postaci listu
nie jest uważany za całkowicie bezpieczny.

\subsubsection{SMS}
Obecnie najczęściej spotykanym kanałem dostarczania haseł jednorazowych są wiadomości SMS (Short Message Service). 
W momencie wykonywania akcji wymagającej dodatkowego uwierzytelnienia, na telefon komórkowy użytkownika 
wysyłane jest pojedyncze hasło jednorazowe. \\
Pomijając koszty związane z wysyłaniem wiadomości SMS, kanał ten posiada zastrzeżeń w kwestii bezpieczeństwa.
Standard szyfrowania \textit{A5/x} używany w wiadomościach został wielokrotnie złamany a wady protokołu 
\textit{SS7 (Signaling System No. 7)}, odpowiedzialnego za przekierowywanie wiadomości pozwalają na 
przekierowanie konkretnej wiadomości do atakującego. \\
W lipcu 2016 roku \textit{Narodowy Instytut Standaryzacji i Technologii} wydał oświadczenie, w~którym
zalecane jest nieużywanie tego kanału przy implementacji uwierzytelnienia dwuetapowego.

\subsubsection{Aplikacja mobilna}
Możliwe jest też wygenerowanie haseł jednorazowych korzystając z aplikacji mobilnych 
dostępnych na smartfonach. Kod źródłowy takich aplikacji zwykle jest otwarty, dzięki czemu 
łatwo jest sprawdzić czy hasła generowane w aplikacji nie są wysyłane gdzieś poza telefon. 
Aplikacje te działają bez dostępu do sieci komórkowej czy do Internetu. \\
Bezpieczeństwo tego kanału jest silnie zależne od stanu bezpieczeństwa samego telefonu. 
Przykładowo blokada ekranu za pomocą silnego hasła zabezpiecza również dostęp do aplikacji generującej
hasła jednorazowe. \\ \\
Istnieje duża liczba aplikacji mobilnych, które można tutaj wykorzystać:
\begin{itemize}
	\item \textit{Google Authenticator}
	\item \textit{Microsoft Authenticator}
	\item \textit{Authy 2-Factor Authentication}
	\item \textit{Duo Mobile}
	\item \textit{HDE OTP}
	\item \textit{Authenticator Plus}
	\item \textit{FreeOTP}
\end{itemize}

\subsubsection{Token sprzętowy}
Token jest niewielkim urządzeniem z wyświetlaczem, na którym pojawiają się kolejne hasła jednorazowe. \\
Nieco bardziej bezpieczne tokeny oprócz wyświetlacza posiadają jeszcze klawiaturę numeryczną. 
Wymagane jest wtedy podanie PINu przed uzyskaniem hasła jednorazowego. 
Zgubienie lub kradzież tokenu nie jest wtedy problemem. \\
Tokeny sprzętowe pomimo posiadania wysokiego poziomu bezpieczeństwa wiążą się z dosyć wysokimi kosztami
użytkowania w porównaniu do aplikacji mobilnych. Pojawia się również problem synchronizacji zegara 
w przypadku haseł opartych o czas.

\subsubsection{Ołówek i kartka papieru}
Pomimo, że może wydawać się to niedorzeczne, możliwe jest wygenerowanie hasła jednorazowego, opartego na liczniku, 
przy użyciu wyłącznie ołówka i kartki papieru. 
Z badań przeprowadzonych przez Kena Shirriffa \cite{pandp} wynika, że obliczenie jednego kroku funkcji skrótu SHA-256
zajmuje około 16 minut i 45 sekund. Używając konstrukcji HMAC funkcja skrótu jest używa dwa razy.
Nietrudno więc policzyć że sam proces obliczania wyniku funkcji HMAC wyniósł by około $1.49$ dnia. \\
Nie jest to efektywny sposób, lecz jeśli nie ufamy wszystkim poprzednim metodom i dysponujemy 
wystarczającą ilością wolnego czasu, możemy wygenerować hasło jednorazowe bez pomocy zewnętrznych 
systemów obliczeniowych.

\section{Interfejs Windows Data Protection}
,,Jeśli wpisujesz do swojego kodu źródłowego litery A-E-S to robisz coś źle'' \cite{lettersaes}. 
Używając kryptografii zalecane jest pracować na jak najwyższym poziomie abstrakcji jaki tylko 
jest dostępny. Im głębiej wchodzi się w detale implementacyjne tym większa szansa, że 
gdzieś zostanie popełniony błąd. A w tej dziedzinie, jeden błąd wystarczy by zaburzyć 
bezpieczeństwo całego systemu kryptograficznego. \\
\textit{Windows Data Protection} jest prostym, wysokopoziomowym interfejsem dostępnym w systemach 
\textit{Windows}, począwszy od wersji \textit{Windows 2000}. 
Podstawowym jego zadaniem jest zarządzanie kluczami kryptograficznymi, poprzez szyfrowanie symetryczne,
wykorzystując klucze systemowe. \\ 
Z każdą kolejną wersją systemu \textit{Windows} interfejs \textit{Interfejs Windows Data Protection} 
jest wykorzystywany coraz powszechniej, obecnie wykorzystywany jest między innymi w takich miejscach jak:
\begin{itemize}
	\item \textit{Microsoft Outlook} przy standardzie \textit{S/MIME}
	\item \textit{Credential Manager}
	\item Przestrzeń \textit{Microsoft.Owin} 
	\item \textit{Windows Live Mail}
	\item Hasła Zdalnego Pulpitu
	\item Hasła Sieci bezprzewodowych
	\item Szyfrowanie bazy danych \textit{Microsoft SQL Server} 
	\item Szyfrowany system plików EFS (Encrypting File System)
\end{itemize}
Bezpieczeństwo danych zaszyfrowanych tym interfejsem w znacznym stopniu zależne jest od hasła 
systemu \textit{Windows}. W przypadku ujawnienia tego hasła lub ataku na bazę kont użytkowników system,
zaszyfrowane dane narażone są na możliwe ataki. \\
Używane algorytmy kryptograficzne różnią się w zależności o wersji systemu operacyjnego. 
W systemie operacyjnym \textit{Windows 7} do szyfrowania symetrycznego wykorzystywane jest szyfr \textit{AES} 
w trybie \textit{CBC} z kluczem o długości 256 bitów. 
Do generowania skrótów używana jest funkcja \textit{SHA512} \cite{dpapi}.
