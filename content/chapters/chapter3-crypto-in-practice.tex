\chapter{Kryptografia w praktyce}

\section{Pojęcia pomocnicze}
Przed przystąpieniem do opisu praktycznych aspektów kryptografii użytych w projekcie, wymagane jest 
wyjaśnienie pojęć wykorzystywanych w mechanizmie haseł jednorazowych, lecz które nie są bezpośrednio 
związane z kryptografią. 

\subsection{Kodowanie transportowe}
Kodowanie transportowe wykorzystywane jest w przypadku, gdy zachodzi potrzeba transferu danych
w środowiskach, które pozwalają na przesyłanie wyłącznie znaków ASCII. \\ \\
Użycie kodowania transportowego jest konieczne w celu zachowania kompatybilności przy pracy z protokołami, 
które przystosowane są do pracy na danych 7-bitowych. W takim przypadku najstarszy bit jest zerowany, co
mogłoby uszkodzić przesyłane dane. W przypadku przesyłania wyłącznie znaków ASCII zerowanie najstarszego bitu
nie jest problemem, gdyż wszystkie znaki w podstawowej tablicy ASCII mają ten bit wyzerowany.\\
Bardziej współczesnym przykładem wykorzystania kodowania transportowego jest osadzanie danych graficznych bezpośrednio w kodzie HTML. Konieczne jest wówczas zakodowanie danych w celu wyeliminowania ryzyka pojawienia się znaków '<' oraz '>', które mogłyby być zinterpretowane jako tagi HTML. \\ \\
Aby ujednolicić implementacje kodowania transportowego został stworzony dokument RFC 4648 \cite{encoding}, w którym opisany jest prawidłowy sposób implementacji oraz to jaki typ kodowania wybrać w zależności od nałożonych wymagań.

\subsubsection{Kodowanie Base64}
Najczęściej spotykanym typem kodowania transportowego jest kodowanie Base64. Kodowanie to konwertuje dowolny ciąg bajtów do postaci ciągu złożonego z małych i wielkich liter, cyfr oraz znaków '+' i '/'. 
Jeżeli po zakodowaniu końcowa część danych jest mniejsza niż 24 używany jest także znak '=' jako dopełnienie. \\
Sam proces kodowania polega na pobraniu 24 bitów danych a następnie podzieleniu ich na 4 grupy po 6 bitów. Każda z grup jest interpretowana jako indeks tablicy ustalonego alfabetu Base64.
Dla każdej z grup za pomocą indeksu odczytywany jest znak a następnie dopisywany jest on do ciągu zakodowanego. \\ \\

Istnieje również odmiana kodowania Base64 przystosowana do użycia w przypadku adresów URL czy nazw plików.
W alternatywie tej zamiast znaków '+', '/', które mogłyby zostać błędnie zinterpretowane np w środowisku systemu plików,
używane są znaki '-' oraz '_'.

\subsubsection{Kodowanie Base32}

\subsection{Czas uniksowy}
Czas uniksowy jest sposobem na reprezentację punktu w czasie, polegającym na mierzeniu sekund, które
upłynęły od daty 1 stycznia 1970 (UTC). W systemach uniksowych zwykle reprezentowany jest w postaci
32-bitowej liczby całkowitej ze znakiem. \\
W przypadku architektur typu serwer-klient wskazane jest synchronizowanie czasu wykorzystując czas~uniksowy,
gdyż nie zależy on od lokalizacji w której jest mierzony. Właściwość ta eliminuje problem synchronizacji 
czasu pomiędzy strefami czasowymi.

\subsection{Ujednolicony identyfikator zasobów}
Ujednolicony identyfikator zasobów (ang. Uniform Resource Identifier, URI) jest ciągiem znaków jednoznacznie identyfikującym dany zasób. \\
Składnia identyfikatora jest wyrażana następująco: \\
\centerline{schemat ":" ścieżka ["?" zapytanie] ["\#" fragment]} 
Warto zauważyć, że składnia ta determinuje schemat (protokół), jaki wykorzystywany jest przy interakcji z identyfikowanym zasobem. \\

Przykłady identyfikatorów:
\begin{itemize}
	\item ftp://randomftp.com/files/file.docx
	\item https://www.randomwebsite.pl/index.html
	\item mailto:jan.nowak@wp.pl
	\item tel:+48-25-123-88
\end{itemize}

Szczegóły dotyczące standardu URI są opisane w dokumencie RFC~3986 \cite{uri}. 

 
\section{Hasło jednorazowe}

\section{Interfejs Windows Data Protection}
