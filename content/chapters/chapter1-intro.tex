\chapter{Wstęp}

\section{Pojęcie uwierzytelnienia wielopoziomowego}
Do uwierzytelnienia użytkownika można wykorzystać następujące czynniki:
\begin{enumerate}
	\item To co użytkownik wie, na przykład hasło lub kod PIN.
	\item To co użytkownik posiada, na przykład fizyczny klucz, token, telefon, karta magnetyczna, identyfikacyjna karta elektroniczna.
	\item To kim użytkownik jest, na przykład odcisk palca, tęczówka oka.
\end{enumerate}
Uwierzytelnienie wielopoziomowe polega na wykorzystaniu więcej niż jednego czynnika w~procesie potwierdzania tożsamości użytkownika.
Uwierzytelnianie dwuetapowe (zwane również dwuetapową weryfikacją) jest metodą wykorzystującą dwa czynniki uwierzytelniające. \\
Przykładowo podczas logowania użytkownik oprócz tradycyjnego hasła (coś co użytkownik wie) jest proszony także
o podanie hasła jednorazowego, wygenerowanego na tokenie sprzętowym (coś co użytkownik posiada) lub 
w przypadku wypłaty bankomatowej karta magnetyczna jest czymś co użytkownik posiada, a kod PIN czymś co użytkownik wie.

\section{Korzyści płynące z używania uwierzytelnienia wielopoziomowego}
Użycie uwierzytelniania wieloetapowego drastycznie zmniejsza ryzyko kradzieży kont użytkowników, gdyż 
wejście w posiadanie hasła użytkownika nie wystarczy by uzyskać dostęp do jego konta.
Uwierzytelnienie dwuetapowe pozwala więc stosunkowo niskim koszem znacznie zwiększyć bezpieczeństwo kont użytkowników. \\
Pomimo, że istnieje standard określający sposób implementacji uwierzytelniania wielopoziomowego to wciąż jednak brakuje oprogramowania, 
które byłoby zgodne ze standardem, będąc przy tym otwarte i proste w użyciu. 

\section{Cel pracy}
Celem pracy jest implementacja platformy umożliwiającej dwuetapową weryfikację jak również scharakteryzowanie 
elementów kryptografii, na których dwuetapowa weryfikacja bazuje. 
W sytuacji, gdy jakiś serwis zechce wykorzystać uwierzytelnianie dwuetapowe,
zamiast implementować cały protokół samodzielnie, może wykorzystać platformę, będącą tematem pracy. \\
Głównymi założeniami projektu jest prostota w użyciu oraz możliwość wykorzystania jego funkcjonalności 
bez względu na używaną technologię czy język programowania.