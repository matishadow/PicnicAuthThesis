\chapter{Zakończenie}

\section{Podsumowanie i wnioski}
Prowadząc serwis, w którym użytkownicy mają możliwość tworzenia kont, jesteśmy odpowiedzialni za ich bezpieczeństwo. 
Poziom tego bezpieczeństwa można podnieść wykorzystując mechanizm dwuetapowej weryfikacji. 
W dzisiejszych czasach ataki na konta użytkowników są tak częste, że zabezpieczenie konta wyłącznie tradycyjnym hasłem już nie wystarcza. \\
Zaimplementowany projekt \textit{PicnicAuth} pozwala na wdrożenie w dowolnym serwisie mechanizmu dwuetapowej weryfikacji, 
w niezwykle prosty i bezpieczny sposób. 
Do skorzystania z funkcjonalności projektu nie jest potrzebna specjalistyczna wiedza z dziedziny kryptografii 
a~otwarty kod projektu pozwala na weryfikację standardu bezpieczeństwa projektu.

\section{Podziękowania}
Chciałbym w tym miejscu podziękować kilku osobom, bez których ta praca by nie powstała:
\begin{itemize}
	\item Promotorowi, dr. hab. Alexanderowi Prokopenya.
	\item Karolinie Wasilewskiej oraz Dominikowi Ostrowskiemu (za wspólne projekty).
\end{itemize} 