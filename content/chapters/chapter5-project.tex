\chapter{PicnicAuth}

\section{Architektura projektu}
Projekt składa się z trzech komponentów: serwera w architekturze REST (Representational state transfer) API, 
informacyjnej strony internetowej oraz bibliotek klienckich. Komponenty te są od siebie niezależne, dzięki czemu 
łatwa jest rozbudowa projektu, jak również wersjonowanie każdego z nich.

\subsection{REST API}
Pierwszym z nich jest serwer implementujący logikę aplikacji. 
Jest on stworzony w technologii \textit{ASP.NET Web API 2} w języku programowania \textit{C\#}. \\
Przy jego tworzeniu zostały zachowane zasady architektury REST, co pozwala na łatwą i sprawną integracją
platform klienckim, niezależnie od użytej technologii. \\ \\
Zasoby udostępniane przez REST API to:
\begin{itemize}
	\item \textit{[GET] /api/Companies/Me/AuthUsers} \\
		Zwraca listę użytkowników dla aktualnie zalogowanego podmiotu.
	\item \textit{[POST] /api/AuthUsers} \\
		Tworzy nowego użytkownika i dodaje go do kolekcji użytkowników zalogowanego podmiotu.
		W odpowiedzi zwracany jest sekret stworzonego użytkownika, jak również link 
		do kodu QR kompatybilnego z aplikacjami mobilnymi.
	\item \textit{[PATCH] /api/AuthUsers/{userId}/secret} \\
		Generuje nowy sekret dla użytkownika o podanym \textit{userdId}.
	\item \textit{[GET] /api/Companies/Me} \\ 
		Zwraca dane zalogowanego podmiotu, takie jak login, adres poczty elektronicznej oraz unikalny identyfikator.
	\item \textit{[POST] /api/Companies} \\
		Tworzy nowe konto podmiotu. 
	\item \textit{[GET] /api/AuthUsers/{userId}/hotp} \\
		Zwraca hasło jednorazowe typu \textit{HOTP} dla użytkownika o podanym \textit{userId}.
	\item \textit{[GET] /api/AuthUsers/{userId}/totp} \\
		Zwraca hasło jednorazowe typu \textit{TOTP} dla użytkownika o podanym \textit{userId}.
	\item \textit{[GET] /api/AuthUsers/{userId}/hotp/{hotp}} \\
		Zwraca wynik weryfikacji podanego hasła jednorazowego typu \textit{HOTP}
	\item \textit{[GET] /api/AuthUsers/{userId}/totp/{totp}} \\
		Zwraca wynik weryfikacji podanego hasła jednorazowego typu \textit{TOTP}
	\item \textit{[POST] /api/tokens}
		Zwraca klucz API, który służy do uwierzytelnienia podmiotu. (Równoznaczne z logowaniem.)
\end{itemize}
Aplikacja przeznaczona jest do działania na serwerze \textit{IIS (Internet Information Service)} w~wersji~7.

\subsection{Frontend}
Strona internetowa projektu stworzona została w technologii \textit{Angular} w wersji~5 w konwencji 
\textit{Single Page Application}. Znajduje się na niej instrukcja użycie projektu jak również informacje
o aktualnej liczbie dostępnych bibliotek. Zawiera ona także przydatne odnośniki do repozytoriów, w których
znajduje się kod źródłowy oraz do strony na której znajduje się dokumentacja API. \\
Dodatkowymi funkcjonalnościami jakie oferuje strona jest stworzenie nowego konta dla podmiotu, używającego 
projektu, jak również uzyskanie klucza API, umożliwiającego użycie bibliotek klienckich.

\subsection{Biblioteki klienckie}
W celu ułatwienia integracji z serwerem zaimplementowane zostały biblioteki klienckie.
Umożliwiają one w prosty sposób użycie funkcjonalności serwera bez konieczności 
studiowania dokumentacji i pisania kodu odpowiedzialnego za integrację. \\
Przykładowo biblioteka w języku C\# udostępnia klasę \textit{PicnicAuthClient}, posiadająca metody:
\begin{itemize}
	\item Login
	\item GetAuthUsers
	\item AddAuthUser
	\item GenereteNewSecret
	\item GetLoggedCompany
	\item AddCompany
	\item GetHotpForAuthUser
	\item ValidateHotpForAuthUser
	\item GetTotpForAuthUser
	\item ValidateTotpForAuthUser
\end{itemize}
Na chwilę obecną gotowe do użycia są biblioteki w następujących technologiach:
\begin{enumerate}
	\item C\#
	\item Visual Basic
	\item TypeScript
	\item Python 3.6
	\item Python 2.7
	\item Ruby
\end{enumerate}

\section{Dostarczanie sekretu na urządzenie mobilne}
\section{Generowanie OTP po stronie serwera}
\section{Generowanie OTP po stronie użytkownika}
\section{Przechowywanie sekretu użytkownika}
\section{Przykład użycia projektu}
\section{Planowane ulepszenia}

